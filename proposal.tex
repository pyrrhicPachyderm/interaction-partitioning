\documentclass[a4paper,11pt]{article}
\usepackage{amsmath}
\usepackage{siunitx}
\usepackage{import}

%Referencing
\import{reference-styles/}{authoryear.tex}
\addbibresource{bibliography/references.bib}
\usepackage{placeins} %Provides \FloatBarrier

%Title Page
\title{BIOL424 Project Proposal}
\author{
	Christopher Brown\\15413822
}
\date{}

\begin{document}
\maketitle

\section{Background}

The simplest standard model of community dynamics is the Lotka--Volterra model, and this is frequently analysed to answer questions of species coexistence, community stability, and the like.
It is parameterised by a vector of intrinsic growth rates, and a matrix of interaction coefficients.
But even for such a simple model, this leads to the number of parameters scaling quadratically with the number of species: $n(n+1)$ parameters for $n$ species.
This makes it infeasible to experimentally parameterise the model for realistically-sized communities.
Can we do better?

Species come in groups: this is the basis of taxonomy.
And we might expect that species in the same group interact in similar ways, either in their effect on other species, or their response to them.
This, in turn, may reduce the number of parameters we require.

\section{Question}

\section{Hypotheses}

\section{Related Literature}

\section{Novelty}

\section{Method Outline}

\FloatBarrier
\printbibliography
\end{document}
