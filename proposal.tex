\documentclass[a4paper,11pt]{article}
\usepackage{amsmath}
\usepackage{siunitx}
\usepackage{import}

%Referencing
\import{reference-styles/}{authoryear.tex}
\addbibresource{bibliography/references.bib}
\usepackage{placeins} %Provides \FloatBarrier

\usepackage{hyperref}

%Title Page
\title{BIOL424 Project Proposal}
\author{
	Christopher Brown\\15413822
}
\date{}

\begin{document}
\maketitle

\section{Background}

The simplest standard model of community dynamics is the Lotka--Volterra model, and this is frequently analysed to answer questions of species coexistence, community stability, and the like.
It is parameterised by a vector of intrinsic growth rates, and a matrix of interaction coefficients.
But even for such a simple model, this leads to the number of parameters scaling quadratically with the number of species: $n(n+1)$ parameters for $n$ species.
This makes it infeasible to experimentally parameterise the model for realistically-sized communities.
Can we do better?

Species come in groups: this is the basis of taxonomy.
And we might expect that species in the same group interact in similar ways, either in their effect on other species, or their response to them.
This, in turn, may reduce the number of parameters we require.

\section{Question}

Do species form groups that are sufficiently similar in their interaction coefficients that they might be considered equivalent in parameterising a model?
If so, how?

\section{Hypotheses}

\begin{enumerate}
	\item There will be some grouping of species such that a Lotka--Volterra model will fit better when parameterised with that grouping than when parameterised entirely ungrouped.
	\item There will be no substantial benefit to model fit from grouping competitive effect and response separately, i.e. it will be sufficient to use the same groups for the rows as for the columns of the interaction matrix.
	\item Some grouping informed solely by phylogeny, at some taxonimic level or similar, will prove near-optimal among groupings.
\end{enumerate}

\section{Related Literature}

In no particular order:
\begin{itemize}
	\item
		\Textcite{allesina2009food} addressed the optimal grouping of species in order to reduce the number of parameters in a model.
		However, they deal with food webs, rather than competitive interactions, and with binary links, rather than quantitative interactions.
	\item
		\Textcite{maynard2020predicting} provide a method of predicting the outcomes of community dynamics given the results of smaller set of experiments than would traditionally be required.
		However, they do not parameterise a model in doing so, and can predict only the equilibrium points of assemblages.
	\item
		\Textcite{uriarte2004spatially} is one of a number of studies that have attempted to group species using groups selected \emph{a priori} based upon functional or phylogenetic factors.
		In particular, the authors of this paper consider and compare both functional and phylogenetic groupings.
		However, they still only consider four groupings, selected \emph{a priori}, out of more than a googol possible groupings.
	\item
		\Textcite{ovaskainen2017species} fitted models of community dynamics that were not parameterised directly by interaction coefficients, but by the responses of species to ``community-level drivers'': weighted averages of the abundances of various species in the community.
		This is a similar technique to grouping (and in fact includes grouping as a subcase where weights are restricted to zero and one), but requires \emph{a priori} definition of the number of community-level drivers.
		Further, its interpretation may be rather distinct from grouping, depending upon results.
	\item
		\Textcite{weiss2021disentangling} address the problem of too many parameters by making the assumption that many of the interaction coefficients are zero.
		Precisely which coefficients are zero is selected on the basis of model fit, rather than \emph{a priori}.
		However, this still assumes that many pairs of species do not interact: an assumption that need not be made by a model based on grouping.
\end{itemize}

\section{Novelty}

As far as I am aware, my study will constitute the first attempt to select the optimal grouping, from among all possible groupings, of a competitive interaction matrix based on quality of model fit.

\section{Method Outline}

\FloatBarrier
\printbibliography
\end{document}
