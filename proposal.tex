\documentclass[a4paper,11pt]{article}
\usepackage{amsmath}
\usepackage{siunitx}
\usepackage{import}

%Referencing
\import{reference-styles/}{authoryear.tex}
\addbibresource{bibliography/references.bib}
\usepackage{placeins} %Provides \FloatBarrier

\usepackage{hyperref}

%Title Page
\title{BIOL424 Project Proposal}
\author{
	Christopher Brown\\15413822
}
\date{}

\begin{document}
\maketitle

\section{Background}

The simplest standard model of community dynamics is the Lotka--Volterra model, and this is frequently analysed to answer questions of species coexistence, community stability, and the like.
It is parameterised by a vector of intrinsic growth rates, and a matrix of interaction coefficients.
But even for such a simple model, this leads to the number of parameters scaling quadratically with the number of species: $n(n+1)$ parameters for $n$ species.
This makes it infeasible to experimentally parameterise the model for realistically-sized communities.
Can we do better?

Species come in groups: this is the basis of taxonomy.
And we might expect that species in the same group interact in similar ways, either in their effect on other species, or their response to them.
This, in turn, may reduce the number of parameters we require.

\section{Question}

Do species form groups that are sufficiently similar in their interaction coefficients that they might be considered equivalent in parameterising a model?
If so, how?

\section{Hypotheses}

\begin{enumerate}
	\item There will be some grouping of species such that a Lotka--Volterra model will fit better when parameterised with that grouping than when parameterised entirely ungrouped.
	\item There will be no substantial benefit to model fit from grouping competitive effect and response separately, i.e. it will be sufficient to use the same groups for the rows as for the columns of the interaction matrix.
	\item Some grouping informed solely by phylogeny, at some taxonimic level or similar, will prove near-optimal among groupings.
\end{enumerate}

\section{Related Literature}

\section{Novelty}

\section{Method Outline}

\FloatBarrier
\printbibliography
\end{document}
